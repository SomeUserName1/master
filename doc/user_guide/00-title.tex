\begin{titlepage}
\thispagestyle{plain}
   \begin{center}
       \vspace*{1cm}
        \Huge
       \textbf{Graph Record Layout Research Environment} \\
       \Large
       \textbf{A graph database with some layout tools \& methods} \\
       \vspace*{1cm}
       \normalsize
       Fabian Klopfer \vspace{1em}\\
       Databases and Information Systems Group \\
       Department of Computer Science \\
       University of Konstanz\\
       \includegraphics[keepaspectratio, height=0.3\textheight]{img/logo.pdf}
        \vspace*{1cm}
        
        \textbf{Abstract: \\}
        The here described software is focussed on finding new methods to layout the records of a graph database such that the least possible accesses are necessary to satisfy a certain benchmark. A benchmark is here usualy a set of queries, for example traversal-based queries. The current state-of-the-art methods use the graph structure to statically layout the graph on disk using clustering, partitioning or community detection methods to form blocks along with some ordering algorithm.
        The methods that are to be explored are dynamic --- either in terms of the query or in terms of a changing database.
        This document provides a high-level user guide to the structure of the database and the methods and tools that are implemented. Furthermore the specification as well as a more extensive design document are captured, along with some benchmark and visualization techniques.
   \end{center}
\end{titlepage}

\newpage
\tableofcontents
