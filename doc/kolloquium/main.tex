 \documentclass[rgb]{beamer}
\usepackage{silence,lmodern}
\usepackage[backend=biber, style=nature]{biblatex}
\usepackage{csquotes}
\usepackage{listings}
\usepackage{svg}
\usepackage{multirow}
\usepackage{booktabs}
\usepackage{multicol}
\usepackage{forest}
\usepackage[format=plain, justification=justified, labelfont=bf, textfont=it, font=small, labelsep=space]{caption}
\usepackage{appendixnumberbeamer}

\WarningFilter{biblatex}{Patching footnotes failed}
\WarningFilter{etex}{Extended allocation already in use}

\renewcommand*{\bibfont}{\tiny}
\bibliography{resources.bib}

\usetheme{Konstanz}
\setcounter{secnumdepth}{3}
\format{169}


\title{Locality Optimization for traversal-based Queries on Graph Databases}
\titleCorporateDesign{Locality Optimization}{for traversal-based}{Queries on}{Graph Databases}
\author{Fabian Klopfer} 
\date{30.04.2021}
\institute{Databases and Information Systems Group \\ Department of Computer and Information Science \\ University of Konstanz}

\AtBeginDocument{
\usebeamerfont{normalfont}
\begin{frame}
	\titlepage
\end{frame}
}

\begin{document}
\section{Introduction}
        \begin{frame}{Motivation}
        Current state of performance-optimized \\ [0.5em]
        \begin{itemize}
            \item relational databases: accesses are made as sequential as possible. \\
            [1em]
            
            \item graph databases: access is often random.
        \end{itemize}
    \end{frame}
    
    \begin{frame}[allowframebreaks,fragile]{Example}
        Show me all boats owned by sailors with an ID less than 50:
        
        \begin{figure}[htp]
        \begin{center}
        \begin{forest}
            [ $\quad \bowtie^m$
            [ $\quad \bowtie^m$
                [ index $\sigma_{\text{ID} < 50}$ S ]
                [ scan O ]
            ]
            [ scan B ]
        ]
        \end{forest} 
        \end{center}        
        \end{figure}
        Reads are mostly sequential. \\   
        $\Rightarrow$ Prefetch \& cache hit.
        
        \framebreak
        
        Nodes are Sailors and Boats, relationships ``owns''
        
        \begin{figure}[htp]
        \begin{center}
          \begin{forest}
            [ $\sigma_{y.\text{label = Boat}}$
            [ $\uparrow_x^y$
                [ $\sigma_{x.ID < 50}$ ($\bigcirc$ x) ]
            ]
        ]
        \end{forest} 
        \end{center}        
        \end{figure}
        Scaning and filtering is sequential. \texttt{Expand} is not. \\ 
        $\Rightarrow$ \texttt{Expand} causes prefetch \& cache misses.
                
            \framebreak
            
            \begin{itemize}
            \item Especially \texttt{Expand} jumps a lot. Potentially back and forth. \\ [2em]
            \item Traversals rely primarily on expand. 
            \end{itemize}
    \end{frame}
        
    \section{Background}
        \begin{frame}{Property Graph Model}
            \begin{figure}
                \begin{center}
                \includegraphics[keepaspectratio, height=0.8\textheight, width=.8\textwidth]{img/property_graph_elements.png}
                \end{center}
            \end{figure}
        \end{frame}
        
         \begin{frame}[allowframebreaks]{Data Structures}
            Two essential record structures: \\ [2em]
            \begin{enumerate}
            \item Node records \\ [2em]
            \item Relationship records \\ [3em]
            \end{enumerate}
            Inspired by Neo4J
            
            \framebreak
            \begin{figure}
                \begin{center}
                \includegraphics[keepaspectratio, height=\textheight, width=\textwidth]{img/node_record.png}
                \end{center}
            \end{figure}
            
            \framebreak
            \begin{figure}
                \begin{center}
                \includegraphics[keepaspectratio, height=\textheight, width=\textwidth]{img/relationship_record.png}
                \end{center}
            \end{figure}
            
            \framebreak
            \begin{figure}
                \begin{center}
                \includegraphics[keepaspectratio, height=0.6\textheight, width=.6\textwidth]{img/graph.png}
                \end{center}
            \end{figure}
            \end{frame}
            
            \begin{frame}
            \vspace{-3.5em}
            \begin{figure}
                \begin{center}
                \includegraphics[keepaspectratio, height=1.2\textheight, width=\textwidth]{img/example_structs.png}
                \end{center}
            \end{figure}
        \end{frame}
        
        \begin{frame}{Graphs we focus on}
            \begin{figure}
                \begin{center}
                \includegraphics[keepaspectratio, height=0.8\textheight, width=0.6\textwidth]{img/data_struct_gr.png}
                \end{center}
            \end{figure}
        \end{frame}
    
        \begin{frame}{Traversals}
            \begin{figure}
                \begin{center}
                \includegraphics[keepaspectratio, height=0.8\textheight, width=0.75\textwidth]{img/bfs-dfs.png} \hfill
                \includegraphics[keepaspectratio, height=0.8\textheight, width=0.2\textwidth]{img/astar.png}
                \end{center}
            \end{figure}
        \end{frame}
                    
    \section{Problem Definition}
        \begin{frame}[allowframebreaks]{Problem Definition}
            Given a graph $G$, logical block size $b$, page size $p$. \\ [2em]
            Desired is \\ [1em]
            \begin{enumerate}
            \item A partition of $G$ into blocks of vertex records $V_i$ and $E_i$ relationship records, \\ [1em]
            \item permutations $\pi_v, \pi_e$ of the blocks of vertex and edge records $V_i, E_i$,\\ [1em]
            \item a reordering of the incidence list pointers \\ [1.5em]
            \end{enumerate}
            such that spatial locality is as high as possible for traversal-based queries.
            
            \framebreak
            
             Temporal locality based on blocks.
                \[ P (X_{t + \Delta} = B | X_t = B) \]
            Spatial locality in the same sense:
                \[ P(X_{t + \Delta} = B \pm \varepsilon | X_t = B) \]
          
            \framebreak
            \begin{figure}[htp]
                \begin{center}
                    \includegraphics[keepaspectratio,height=0.4\textheight,width=\textwidth]{img/example_graph.png}
                \end{center}
            \end{figure}
            
             \begin{table}[htp]
                \centering
                \begin{tabular}[c]{|l|c|c|c|c|c|c|} \hline
                &&&&&&\\[-1em]
                node.db & \colorbox{blue!30}{0}, \colorbox{red!30}{5}, \colorbox{green!30}{7} & \colorbox{blue!30}{1}, \colorbox{blue!30}{4}, \colorbox{green!30}{9} & \colorbox{blue!30}{2}, \colorbox{red!30}{6}, \colorbox{green!30}{8} & \colorbox{blue!30}{3} &  & \\ \hline
                &&&&&&\\[-1em]
                edge.db & \colorbox{blue!30}{a}, \colorbox{green!30}{f} & \colorbox{blue!30}{b}, \colorbox{green!30}{g} & \colorbox{blue!30}{c}, \colorbox{green!30}{h} & \colorbox{blue!30}{d}, \colorbox{green!30}{i} & \colorbox{red!30}{e}, \colorbox{green!30}{j} & \colorbox{green!30}{k} \\  \hline
                \end{tabular}
                \vspace{0.5cm}
                
                \begin{tabular}{|l | c | c | c | c | c | c|} \hline
                &&&&&&\\[-1em]
                node.db & \colorbox{green!30}{7},\colorbox{green!30}{8}, \colorbox{green!30}{9} & \colorbox{blue!30}{0}, \colorbox{blue!30}{1}, \colorbox{blue!30}{3} & \colorbox{blue!30}{2}, \colorbox{blue!30}{4}, \colorbox{red!30}{5} & \colorbox{red!30}{6} &  & \\ \hline
                &&&&&&\\[-1em]
                 edge.db &  \colorbox{green!30}{f}, \colorbox{green!30}{h} & \colorbox{green!30}{g}, \colorbox{green!30}{k} & \colorbox{green!30}{i}, \colorbox{green!30}{j} & \colorbox{blue!30}{a}, \colorbox{blue!30}{b} & \colorbox{blue!30}{c}, \colorbox{blue!30}{d} & \colorbox{red!30}{e} \\ \hline
                \end{tabular}
            \end{table}
        \end{frame}
        
    \section{Locality-optimizing Record Layout}
        \begin{frame}[allowframebreaks, fragile]{G-Store}
            \begin{figure}[H]
                \begin{center}
                \includegraphics[keepaspectratio, height=0.8\textheight, width=\textwidth]{img/multilevel.png}\\
                \end{center}
            \end{figure}
            
            \framebreak
            \vfill\vspace{0pt}
            \begin{enumerate}
             \item Coarsening: Heavy-Edge Matching
             \item Turn-around
             \item Uncoarsening
                \begin{enumerate}
                 \item Project
                 \item Reorder
                 \item Refine\\ [3em]
                \end{enumerate}
            \end{enumerate}
             \[ \min \sum_{(u,v) \in E} |\phi(u) - \phi(v)| \] 
             
            \framebreak
            \begin{figure}[H]
                \begin{center}
                \includegraphics[keepaspectratio, height=0.9\textheight, width=\textwidth]{img/g-store.png}\\
                \end{center}
            \end{figure}
            \vfill\vspace{0pt}
        \end{frame}
        
        \begin{frame}[allowframebreaks]{ICBL}
        \vspace{-3.3em}
            \begin{figure}
                \begin{center}
                \includegraphics[keepaspectratio, height=\textheight, width=\textwidth]{img/icbl.png}
                \end{center}
            \end{figure}
            
            \framebreak
            \begin{enumerate}
             \item[I ] Feature extraction: Do $t$ random walks of length $l$. 
             \item[C] Coarse clustering: Adapted K-Means.
             \item[B] Block Formation: Agglomerative hierarchical clustering.
             \item[L] Layout Blocks: Sort blocks and subgraphs
            \end{enumerate}

            
        \end{frame}
        
        \begin{frame}[allowframebreaks]{Louvain Method}
            \begin{figure}
                \begin{center}
                \includegraphics[keepaspectratio, height=0.8\textheight, width=.8\textwidth]{img/louvain.png}
                \end{center}
            \end{figure}
            
            \framebreak
            
            \begin{enumerate}
             \item Initialize all nodes in singleton community.
             \item Merge community into a neighboring community where modularity gain is maximal, until modularity gain is below threshold.
             \item Construct new graph from aggregated communities and go to 1.
            \end{enumerate}
            \vspace{2em}
            \[ \frac{1}{2m} \sum_{u,v \in V} \left( w_{(u, v)} - \frac{w_u w_v}{2m} \right) \cdot \delta (c_u, c_v) \]
            
        \end{frame}
    
        \begin{frame}{Incidence List Rearrangement}
            TODO figure

        \end{frame}
        
    \section{Evaluation}
        \begin{frame}{Setup}
        \begin{itemize}
         \item Queries: BFS, DFS, Dijkstra, A$^*$, ALT. \\ [1em]
         \item Datasets: $[131, 1'134'890]$ nodes, $[764, 2'987'624]$ edges, average degree $[2.6, 25.5]$ \\ [1em]
         \item Domains include biological neural net, E-Mails, Co-authors, Frequent item sets, Comments.
        \end{itemize}
        \end{frame}
        
        \begin{frame}[allowframebreaks]{Results}
            \begin{figure}
                \begin{center}
                \includegraphics[keepaspectratio, height=\textheight, width=.45\textwidth]{img/c_elegans_Block_unsorted_io_comparison.pdf} \hfill
                 \includegraphics[keepaspectratio, height=\textheight, width=.45\textwidth]{img/youtube_Block_unsorted_io_comparison.pdf}
                \end{center}
            \end{figure}
            
            \framebreak
            \begin{figure}
                \begin{center}
                \includegraphics[keepaspectratio, height=0.8\textheight, width=0.45\textwidth]{img/amazon_Block_unsorted_io_comparison.pdf} \hfill
                \includegraphics[keepaspectratio, height=0.8\textheight, width=0.45\textwidth]{img/amazon_Page_unsorted_io_comparison.pdf}
                \end{center}
            \end{figure}

            \framebreak
            \begin{figure}
                \begin{center}
                \includegraphics[keepaspectratio, height=0.8\textheight, width=.45\textwidth]{img/dblp_g-store_alt_block_sil_access_seq.png} \hfill
                \includegraphics[keepaspectratio, height=0.8\textheight, width=.45\textwidth]{img/youtube_louvain_bfs_page_sil_access_seq.png}
                \end{center}
            \end{figure}
            
            \framebreak
            \begin{figure}
                \begin{center}
                \includegraphics[keepaspectratio, height=0.8\textheight, width=\textwidth]{img/dblp_natural_dfs_block_sil_access_seq.png}
                \end{center}
            \end{figure}
        \end{frame}
    
    \section{Conclusion}
        \begin{frame}{Summary}
            \begin{itemize}
                \item Static rearrangement methods decrease number of block accesses. \\
                $\Rightarrow$ increase locality \\ [1em]
                \item Sorting the incidence lists leads to more sequential access sequences. \\ [1em]
                \item Ordering the blocks is crucial for spatial locality.
                \end{itemize}
        \end{frame}
    
        \begin{frame}{Future Work}
        \begin{itemize}
            \item Leiden instead of Louvain \\ [2em]
            \item RCM-based rearrangement \\ [2em]
            \item  Dynamic Rearrangement --- Query-based
        \end{itemize}
        \end{frame}
\end{document}
