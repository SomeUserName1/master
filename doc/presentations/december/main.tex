 \documentclass[rgb]{beamer}
\usepackage{silence,lmodern}
\usepackage[backend=biber, style=ieee]{biblatex}
\usepackage{csquotes}
\usepackage{listings}
\usepackage{svg}
\usepackage{multirow}
\usepackage{booktabs}
\usepackage{multicol}
\usepackage{forest}
\usepackage[format=plain, justification=justified, labelfont=bf, textfont=it, font=small, labelsep=space]{caption}
\usepackage{appendixnumberbeamer}

\WarningFilter{biblatex}{Patching footnotes failed}
\WarningFilter{etex}{Extended allocation already in use}

\renewcommand*{\bibfont}{\tiny}
\bibliography{resources.bib}

\usetheme{Konstanz}
\setcounter{secnumdepth}{3}
\format{169}


\title{Locality optimization for traversal-based queries on graph databases}
\titleCorporateDesign{Locality Optimization for}{traversal-based Queries}{on}{Graph Databases}
\author{Fabian Klopfer}
\date{17.12.2020}
\institute{Database and Information Systems Group \\ University of Konstanz}

\AtBeginDocument{%
\usebeamerfont{normalfont}
\begin{frame}
	\titlepage%
\end{frame}
}

\begin{document}
\section{Motivation}
    \begin{frame}{Motivation}
        \begin{itemize}
            \item Neo4J\@: nodes and relationship stored in insertion order $\Rightarrow$ Random Access to Disk \vspace{0.5cm} \\
            \item Idea: Store nodes of same community close to each other \\ \vspace{0.3cm}

                Then verify if: \\
                \begin{itemize}
                    \item Nodes used frequently together are closer/on the same page \\
                        $\Rightarrow$ less pages must be accessed
                    \item The contents of neighbouring pages are frequently used together \\
                        $\Rightarrow$ pages can be loaded sequentially
                    \item Overall IO is thus reduced
                \end{itemize}
        \end{itemize}
    \end{frame}

\section{Related Work}
    \begin{frame}[allowframebreaks]{Related Work}
        \begin{itemize}
            \item Steinhaus, Olteanu: G-Store~\autocite{steinhaus} \\
                Storage Manager for Graph DB using modified version of Multi-level partitioning~\autocite{karypis} \vspace{0.5cm} \\
            \item Gedik with Ya\c{s}ar and Bordawekar \vspace{0.3cm} \\
                \begin{itemize}
                    \item Diffusion sets and coarse partitioning~\autocite{yacsar,yacsar1}
                    \item Conductance and Cohesiveness metrics, adaptive~\autocite{gedik} \vspace{0.5cm} \\
                \end{itemize}
            \item Pacher, Specht: N-Body Simulation \vspace{0.8cm} \\
        \end{itemize}
        Most of them quantify query runtime or distance \\
    \end{frame}

\section{Current State}
    \begin{frame}{Current state}
        \begin{itemize}
            \item Documented Neo4J record formats in detail \vspace{0.5cm} \\
            \item Implementing: Graph DB Storage Layer with Record \& File layouts mimicing Neo4J layouts \vspace{0.5cm} \\

            \item Implement record rearrangement strategy optimizing locality \vspace{0.5cm} \\

            \item Compare random placement with to be derived method for the ``queries'': BFS, Dijkstra, A$^*$ \vspace{0.5cm} \\
        \end{itemize}
    \end{frame}

\section{The Cost Model}
    \begin{frame}{The Cost Model}
        \begin{itemize}
            \item Runtime-based measurement of own implementation hardly comparable to Neo4J\@: \vspace{0.3cm} \\
                \begin{itemize}
                    \item used language: Java vs.\ C
                    \item Buffer Manager: Record layout transforming page cache vs.\ simple buffer manager
                    \item Query language: Cypher vs.\ direct implementations of BFS, Dijkstra, A$^*$ \vspace{0.8cm} \\
                \end{itemize}
                \alert{How to quantify IO $\Rightarrow$ What is the cost model?} \vspace{0.4cm}\\
                Sufficient to implement GetNodes and Expand and use Hölsch, Grossniklaus~\autocite{holsch}?
        \end{itemize}
    \end{frame}

    \begin{frame}{Bibliography}
        \printbibliography%
    \end{frame}

\end{document}
