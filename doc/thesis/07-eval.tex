\chapter{Experimental Evaluation}\label{\positionnumber} 
\section{Implementation}\label{\positionnumber}
    The implementation is written in C and has currently approximately $10 000$ lines of code.
    It comprises data structures, likehash tables, lists, queues and fibonacci heaps. 
    The database currently only operates in-memory, implementing an access layer using the \mintinline{c}{expand} and \mintinline{c}{get_nodes} operations.
    The record structures are similar to those used in Neo4J. 
    The difference to the structures of Neo4J is that properties, labels and relationship types are currently not supported.
    Besides these differences, the data structures are described as in \ref{n4j}.
    Instead of using two files, two hash tables are used to store the records --- one for nodes and one for edges.
    
    Atop of the access layer, all traversal, shortest path and the louvain algorithm is implemented. These algorithms are augmented to log the IDs of accessed nodes and relationships.
    Further the locality optimizing data layout algorithms of G-Store and ICBL are contained, along with an implementation of the louvain based formation and the RCM-based ordering of blocks.
    Additionally routines for counting the number of blocks that are accessed based on a sequence of accessed nodes and relationhips, an importer for datasets from the SNAP collection and means of sorting the incidence lists are provided.
    
\section{Experimental Setup}\label{\positionnumber}
    \subsection{Data Sets and Queries}\label{\positionnumber}
    SNAP, traversal, shortest path
    \subsection{Environment}\label{\positionnumber}
    MBP, Desktop
    
\section{Results}\label{\positionnumber}
